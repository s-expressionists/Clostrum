\chapter{Compilation environment}

\section{Class}

\Defclass {compilation-environment}

An instance of this class is passed as the \texttt{\&environment}
argument to macro functions during compile time.

\Definitarg {:parent}

This initialization argument must be provided and should be
an evaluation environment.

\Defgeneric {parent} {environment}

\Defmethod {parent} {(environment {\tt compilation-environment})}

This generic function returns the environment objection that was passed
as the \texttt{:parent} initialization argument when
\textit{environment} was created.

\section{Generic functions}

\Defgeneric {function-lambda-list} {client environment \\ function-name}

This generic function returns two values.  The first value is the
lambda list of the function named \textit{function-name}.  The second
value is true if and only if information about the lambda list of the
function named \textit{function-name} exists.  In other words, if no
information about a the lambda list of a function with that name
exists in \textit{environment}, then the values \texttt{nil} and
\texttt{nil} are returned.

This function is used in a Trucler method in order to return
information about the name of a function that is being encountered in
source code.

\Defgeneric {(setf function-lambda-list)} {lambda-list client environment \\
function-name}

This generic function sets the lambda list class of the function
named \textit{function-name}.

This generic function can be called by the expansion of the macros
\texttt{defun} or \texttt{defgeneric}, in the
\texttt{:compile-toplevel} situation of an \texttt{eval-when} form.
Thus, even though the function can not be created at compile time, the
compiler is informed about the lambda list when a reference to the
function is encountered later during compilation.

\Defgeneric {function-class-name} {client environment \\ function-name}

This generic function returns the name of a the class of the function
named \textit{function-name}.  If no information about a
function with that name exists in \textit{environment}, then
\texttt{nil} is returned.

This generic function is typically used in the expansion of the
\texttt{defmethod} macro, in the \texttt{:compile-toplevel} situation
of an \texttt{eval-when} form.  Recall that the MOP function
\texttt{make-method-lambda} is called at compile time by the expansion
of this macro.  In order to pass a class prototype for the
\texttt{generic-function} argument to \texttt{make-method-lambda},
this generic function can be used to determine which class prototype
to use.

\Defgeneric {(setf function-class-name)} {class-name client environment \\
function-name}

This generic function sets the name of a the class of the function
named \textit{function-name}.

Information about the class of a function is useful mostly for generic
functions, but can also be used for ordinary functions and instances
of the class named \texttt{funcallable-standard-object}, defined by
the metaobject protocol.

\Defgeneric {method-class-name} {client environment \\ function-name}

This generic function returns the name of a the class of the methods
on the generic function named \textit{function-name}.  If no information
about a generic function with that name exists in
\textit{environment}, then \texttt{nil} is returned.

\Defgeneric {(setf method-class-name)} {class-name client environment \\
function-name}

This generic function sets the name of a the class of the methods on
the generic function named \textit{function-name}.

MORE GENERIC FUNCTIONS HERE
