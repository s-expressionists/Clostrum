\chapter{Using \sysname{}}

There are several use cases for \sysname{} depending on the type and
level of ambition of the client.

\section{Implementing standard environment functions}

Unfortunately, the \commonlisp{} standard does not contain a separate
chapter dedicated environment functions.%
\footnote{The chapter of the standard
  entitled ``Environment'' has to do with the computational environment
  of the \commonlisp{} system.}
So for example the function \texttt{symbol-function} is specified in
the chapter entitled ``Symbols'', whereas the function
\texttt{fdefinition} is specified in the chapter entitled ``Data and
Control Flow'', even though the two are very much related.

In this section, we describe how a \commonlisp{} implementation could
implement the standard environment functions using a \sysname{}
run-time environment as a first-class global environment.

\section{Compilation environment only}

An existing (perhaps relatively new) \commonlisp{} implementation that
has decided to take advantage of the permission to merge the run-time
environment and the evaluation environment, but that wants a separate
compilation environment can use \sysname{} as follows:

\begin{itemize}
\item It would define an environment class  (say
  \texttt{client:global-environment}) that has
  \texttt{clostrum:run-time-environment} as a superclass.
\item It would implement methods on the generic functions in
  \refSec{sec-run-time-protocol-functions} that trampoline to existing
  implementation-specific functions in the single global environment
  of the client.
\item It would create a constellation consisting of an instance of
  \texttt{client:global-environment} as the parent of an instance of
  \texttt{clostrum:compilation-environment}
\end{itemize}

\section{Compilation and evaluation environment}

An existing \commonlisp{} implementation with a traditional, single
global run-time environment that wants a separate evaluation
environment in order to avoid compile-time side effects to its
run-time environment can use \sysname{} as follows:

\begin{itemize}
\item It would define an environment class  (say
  \texttt{client:run-time-environment}) that has
  \texttt{clostrum:environment} as a superclass.
\item It would implement methods on the generic functions in
  \refSec{sec-run-time-protocol-functions} that trampoline to existing
  implementation-specific functions in the single global environment
  of the client.
\item It would create a class (say
  \texttt{client:evaluation-environment}) as a subclass of
  \texttt{clostrum:evaluation-environment-mixin} and
  \texttt{client:run-time-environment}.
\item It would create a constellation consisting of an instance of
  \texttt{client:run-time-environment}, an instance of
  \texttt{client:evaluation-environment},
  \texttt{clostrum:compilation-environment}
\end{itemize}
